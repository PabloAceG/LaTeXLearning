\documentclass[10pt]{article}

% %%%%%%%%%%%%%%%%%%%%%%%%%%%%%%%%%%%%%%%%%%%%%%%%%%%%%%%%%%%%%%%%%%%%%%%%%%%%%%
% %                                 PACKAGES                                   %
% %%%%%%%%%%%%%%%%%%%%%%%%%%%%%%%%%%%%%%%%%%%%%%%%%%%%%%%%%%%%%%%%%%%%%%%%%%%%%%
% Identifies input coming with an UTF-8 format
\usepackage[utf8]{inputenc}
% Uses 8-bit T1 fonts (Latin) 
\usepackage{setspace}
 \singlespacing
% Arial font.
\usepackage[scaled]{helvet}
\renewcommand\familydefault{\sfdefault}
\usepackage[T1]{fontenc}
\setlength{\parindent}{0.4cm}
% Equations
\usepackage{amsmath}
% Enum Item
\usepackage{enumitem}
% Easy Lists
\usepackage[ampersand]{easylist}
% Tables
\usepackage{booktabs}
\usepackage{longtable}
% Formatting margins
\usepackage{geometry}
 \geometry{
 a4paper,
 }
% Hyperreferences
\usepackage{hyperref}
% Images
\usepackage{graphicx}
% Set images location
\usepackage{wrapfig}
% Position for tables and images
\usepackage{float}
% Blibliography
\usepackage[backend=biber, style=numeric]{biblatex}
 \addbibresource{artAPA.bib}

% %%%%%%%%%%%%%%%%%%%%%%%%%%%%%%%%%%%%%%%%%%%%%%%%%%%%%%%%%%%%%%%%%%%%%%%%%%%%%%
% %                                  TITLE                                     %
% %%%%%%%%%%%%%%%%%%%%%%%%%%%%%%%%%%%%%%%%%%%%%%%%%%%%%%%%%%%%%%%%%%%%%%%%%%%%%%  
\title{
    Edge of Things: The Big Picture on the Integration of Edge, IoT and Cloud in
    a Distributed Computed Enviroment
}

\author{
Pablo Acereda\\
Computer Science Degree\\
}

% %%%%%%%%%%%%%%%%%%%%%%%%%%%%%%%%%%%%%%%%%%%%%%%%%%%%%%%%%%%%%%%%%%%%%%%%%%%%%%  
% %                                 DOCUMENT                                   %
% %%%%%%%%%%%%%%%%%%%%%%%%%%%%%%%%%%%%%%%%%%%%%%%%%%%%%%%%%%%%%%%%%%%%%%%%%%%%%%  
\begin{document}

% Creating title.
\maketitle

% %%%%%%%%%%%%%%%%%%%%%%%%%%%%%%%%% ABSTRACT %%%%%%%%%%%%%%%%%%%%%%%%%%%%%%%%%%%
\begin{abstract}

    As the usage of wireless networks and the Internet of Things (IoT) raise in
    popularity, that involves the risk of latency and traffic in the network.
    With the objective of the suppression of those obstacles the Edge Computing
    (EC) paradigm has been developed. With its integration the processing is
    carried in the edge of the network devices. EC is to increase the response
    time in the applications that previously used the cloud. The scope of this
    article is to prove the efficiency and resourcefulness of EC. As an
    addendum, the EC paradigm is compared with the rest of Cloud Computing
    Systems.

\end{abstract}

% %%%%%%%%%%%%%%%%%%%%%%%%%%%%%%%%% KEYWORDS %%%%%%%%%%%%%%%%%%%%%%%%%%%%%%%%%%%
{\bf Keywords:}
    IoT, 
    cloud computing, 
    edge computing, 
    fog computing, 
    multi-cloud.

\section{Introduction}

\section{Overview of edge computing}

\subsection{Challenges Facing EC}

\section{An Overview of Computing Architecture}

\subsection{Research View on EC}

\subsection{Service Benefits of EC}

\subsection{Computing vs Storage Service of EC/FC/The Cloud/MCC}

\subsection{Computing in Heterogeneous Distributed Networks}

\subsection{Privacy and Security Issues Relating to EC}

\section{Integration of IoT with Edges}

\section{Related Work}

\section{Future Developments on EC}

\section{Conclusion}

% TODO: Delelte the unnecessary elements
% List of elements
%\begin{enumerate}
% \item 
%\end{enumerate}

%\begin{description}[align=left]
% \item [Item1] 
% \item [Item2]
%\end{description}

% \cite{Cited reference}

% Single paged Table 
%\begin{table}[h!]
% \centering
%  \begin{tabular}
%        { | p{3cm} | p{3cm} | p{3cm} | p{3cm} | }
%  \hline
%  Col1 & Col2 & Col3 & Col4 \\
%  \hline
%  Val1 & Val2 & Val3 & Val4 \\
%  \hline
%  \end{tabular}
%\caption{Some caption.}
%\end{table}

%\begin{figure}[h!]
% \centering
 % \includegraphics[width=0.6\textwidth]{nameOfPicture.png}
% \caption{Write a caption.
%          \label{thisIsAReference}}
%\end{figure}

% Bibliography
\section*{Bibliography}
 \printbibliography

\end{document}
